\documentclass[11pt]{article} 
\usepackage[english]{babel}
\usepackage[utf8]{inputenc}
\usepackage[margin=0.5in]{geometry}
\usepackage{amsmath}
\usepackage{amsthm}
\usepackage{amsfonts}
\usepackage{amssymb}
\usepackage[usenames,dvipsnames]{xcolor}
\usepackage{graphicx}
\usepackage[siunitx]{circuitikz}
\usepackage{tikz}
\usepackage[colorinlistoftodos, color=orange!50]{todonotes}
\usepackage{hyperref}
\usepackage[numbers, square]{natbib}
\usepackage{fancybox}
\usepackage{epsfig}
\usepackage{soul}
\usepackage[framemethod=tikz]{mdframed}
\usepackage[shortlabels]{enumitem}
\usepackage[version=4]{mhchem}
\usepackage{multicol}

\usepackage{mathtools}
\usepackage{comment}
\usepackage{enumitem}
\usepackage[utf8]{inputenc}
\usepackage[linesnumbered,ruled,vlined]{algorithm2e}
\usepackage{listings} 
\usepackage{color}
\usepackage[numbers]{natbib}
\usepackage{subfiles}
\usepackage{tkz-berge}


\newtheorem{prop}{Proposition}[section]
\newtheorem{thm}{Theorem}[section]
\newtheorem{lemma}{Lemma}[section]
\newtheorem{cor}{Corollary}[prop]

\theoremstyle{definition}
\newtheorem{definition}{Definition}

\theoremstyle{definition}
\newtheorem{required}{Problem}

\theoremstyle{definition}
\newtheorem{ex}{Example}


\setlength{\marginparwidth}{3.4cm}
%#########################################################

%To use symbols for footnotes
\renewcommand*{\thefootnote}{\fnsymbol{footnote}}
%To change footnotes back to numbers uncomment the following line
%\renewcommand*{\thefootnote}{\arabic{footnote}}

% Enable this command to adjust line spacing for inline math equations.
% \everymath{\displaystyle}

% _______ _____ _______ _      ______ 
%|__   __|_   _|__   __| |    |  ____|
%   | |    | |    | |  | |    | |__   
%   | |    | |    | |  | |    |  __|  
%   | |   _| |_   | |  | |____| |____ 
%   |_|  |_____|  |_|  |______|______|
%%%%%%%%%%%%%%%%%%%%%%%%%%%%%%%%%%%%%%%

\title{
\normalfont \normalsize 
\textsc{CSCI 3104 Spring 2022 \\ 
Instructor: Profs. Chen and Layer} \\
[10pt] 
\rule{\linewidth}{0.5pt} \\[6pt] 
\huge Quiz 3 - Exchange Arguments \\
\rule{\linewidth}{2pt}  \\[10pt]
}
%\author{Your Name}
\date{}

\begin{document}

\maketitle


%%%%%%%%%%%%%%%%%%%%%%%%%
%%%%%%%%%%%%%%%%%%%%%%%%%%
%%%%%%%%%%FILL IN YOUR NAME%%%%%%%
%%%%%%%%%%AND STUDENT ID%%%%%%%%
%%%%%%%%%%%%%%%%%%%%%%%%%%
\noindent
Due Date \dotfill February 4 \\
Name \dotfill \textbf{Julia Troni} \\
Student ID \dotfill \textbf{109280095} \\


\tableofcontents

\section{Instructions}
 \begin{itemize}
	\item The solutions \textbf{should be typed}, using proper mathematical notation. We cannot accept hand-written solutions. \href{http://ece.uprm.edu/~caceros/latex/introduction.pdf}{Here's a short intro to \LaTeX.}
	\item You should submit your work through the \textbf{class Canvas page} only. Please submit one PDF file, compiled using this \LaTeX \ template.
	\item You may not need a full page for your solutions; pagebreaks are there to help Gradescope automatically find where each problem is. Even if you do not attempt every problem, please submit this document with no fewer pages than the blank template (or Gradescope has issues with it).

	\item You \textbf{may not collaborate with other students}. \textbf{Copying from any source is an Honor Code violation. Furthermore, all submissions must be in your own words and reflect your understanding of the material.} If there is any confusion about this policy, it is your responsibility to clarify before the due date. 

	\item Posting to \textbf{any} service including, but not limited to Chegg, Discord, Reddit, StackExchange, etc., for help on an assignment is a violation of the Honor Code.

	\item You \textbf{must} virtually sign the Honor Code (see Section \ref{HonorCode}). Failure to do so will result in your assignment not being graded.
\end{itemize}


\section{Honor Code (Make Sure to Virtually Sign)} \label{HonorCode}

\begin{required}
\noindent 
\begin{itemize}
\item My submission is in my own words and reflects my understanding of the material.
\item I have not collaborated with any other person.
\item I have not posted to external services including, but not limited to Chegg, Discord, Reddit, StackExchange, etc.
\item I have neither copied nor provided others solutions they can copy.
\end{itemize}

%\noindent In the specified region below, clearly indicate that you have upheld the Honor Code. Then type your name. 
\end{required}

\begin{proof}[I agree to the above, Julia Troni.]
%% Typing "I agree to the above," followed by your name is sufficient.
\end{proof}



\newpage
\section{Standard 3- Exchange Arguments}

\subsection{Problem \ref{DFS1}}
\begin{required} \label{DFS1}
Suppose that there are $n$ homework assignments, where the $i$th homework assignment has difficulty $d_{i}>0$. All of the assignments are released on the first day of class, and you may turn in one assignment per week. If you turn in assignment $i$ on week $k$, then you receive $(n-k) e^{d_i}$ points. Do the following.

\begin{enumerate}[label=(\alph*)]
\item Consider a solution in which you turn in assignment $j$ before assignment $i$, even though $d_{i} > d_{j}$. Show that you can increase the number of points earned by turning in assignment $i$ before assignment $j$.

\begin{proof}
%Your answer here.
Suppose we have some solution, $S$, such that you turn in assignment $j$ before assignment $i$ and $d_{i} > d_{j}$. \\
The number of points you earn for $j$ is $(n-k) e^{d_j}$ and for $i$ $(n-(k+s)) e^{d_i}$, where $s \geq 1$ and represents the time between submissions.\\

Now suppose we exchange the order and submit $i$ before $j$ and denote this new solution, $S'$. The number of points you will earn for $i$ is $(n-k) e^{d_i}$. Then for $j$ $(n-(k+s)) e^{d_j}$. Since we know that $d_{i} > d_{j}$, by properties of exponents, it must be that $(n-k) e^{d_i} > (n-(k+s)) e^{d_j}$. Thus, the solution $S'$ will yield more points the solution of $S$. 

Since $i$ and $j$ are arbitrary homework assignments, we can apply this exchange for any two homework assignment, meaning we can submit assignments with higher difficulty before assignments with lower difficult. Thus, we would create $S'$, the ordering of homework assignments to receive the maximum number of points.
\end{proof}


\vskip 50pt
\item Using part (a), describe a greedy algorithm to order the assignments in order to maximize the number of points earned. Pseudo-code is not required, but you should provide enough detail that a CSCI 2270 student could reasonably be expected to implement the solution from your description.

\begin{proof}[Answer]
%Your answer here
A greedy algorithm to maximize the number of points earned is the following:
\begin{enumerate}
\item Use a priority queue to order each homework assignment in order of difficulty such that the first element in the queue is the most difficult, $d_{h} > d_{h+1} > d_{h+2} > ... d_{h+n}$ where $h$ is the index in the queue
\item Submit the most difficult assignment first, then the next most difficult, then then next, until there are no more assignments to submit. So, submit them in order of the priority queue and be sure to submit exactly one each week. 
\end{enumerate}
Hence, by the proof in part $a$, this ordering will maximize the number of points earned
\end{proof}

\end{enumerate}
\end{required}




%%%%%%%%%%%%%%%%%%%%%%%%%%%%%%%%%%%%%%%%%%%%%%%%%%
\end{document} % NOTHING AFTER THIS LINE IS PART OF THE DOCUMENT



